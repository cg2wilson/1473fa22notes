\documentclass[notes]{subfiles}
\begin{document}
	\addcontentsline{toc}{section}{Measurement \& Units}
	\refstepcounter{section}
	\fancyhead[RO,LE]{\bfseries \large \nameref{measurement}} 
	\fancyhead[LO,RE]{\bfseries \currentname}
	\fancyfoot[C]{{}}
	\fancyfoot[RO,LE]{\large \thepage}	%Footer on Right \thepage is pagenumber
	\fancyfoot[LO,RE]{\large Unit 1}

\section*{Measurement \& Units}\label{measurement}
	\subsection*{Measuring Distance}
		There are many ways of measuring distance.  Here are a few common units of measurement for distance.%
		
		\begin{itemize}
			\item Inches (in)%
			\item Feet (ft)%
			\item Yards (yd)%
			\item Miles (mi)%
			\item Centimeters (cm)%
			\item Meters (m)%
			\item Kilometers (km)%
		\end{itemize}

		\begin{ex}
			Give an example of something that can be measured using each of the units above.
		\end{ex}
			\vs{1}
			
		\begin{rmk}[Relationships Between Units]
			\tabitem There are 12 inches in 1 foot\\[15pt]
			\tabitem There are 3 feet in 1 yard\\[15pt]
			\tabitem There are 5280 feet in 1 mile\\[15pt]
			\tabitem There are 100 centimeters in 1 meter\\[15pt]
			\tabitem There are 100 meters in 1 kilometer
		\end{rmk}
			\newpage
			
		\begin{ex}
			Determine an appropriate unit to measure the distance given below.%
			\begin{enumerate}[(a)]
				\item Distance between Norman and Oklahoma City%
					\vs{.5}
					
				\item Height of a tall building%
					\vs{.5}
					
				\item Diameter of a cookie%
					\vs{.5}
					
				\item Length of a bug%
					\vs{.5}
			\end{enumerate}
		\end{ex}
		
		Converting between units is very important to help us conceptualize and compute information.%
		
		\begin{ex}
			If there are 3.28 feet in 1 meter, convert 1 mile to 1 kilometer%
		\end{ex}
			\vs{1}
			\newpage
			
		\begin{ex}
			Convert your height from imperial units (feet and inches) to centimeters.  Use the fact that there are 2.54 centimeters in 1 inch.%
		\end{ex}
			\vs{1}
			
		\begin{ex}
			The average distance between the Earth and the Sun is called an astronomical unit.  If there are 149,600,000 km in 1 AU, how many miles is in 1 AU?
		\end{ex}
			\vs{1}
			
	\subsection*{Measuring Area}
		While length measures one dimensional objects, area is a measure of two-dimensional objects.%
		
		\begin{rmk}[Area Formulas for Common Shapes]
			\tabitem Rectangles: Length\(\cdot\) Width\\[15pt]
			\tabitem Triangles: \(\dfrac{1}{2}\cdot\) Length \(\cdot\) Width\\[15pt]
			\tabitem Circles: \(\pi\cdot\)Radius\(^2\)
		\end{rmk}
		
		The units of area are similar to the units of length; we find the units for area by squaring the component length unit.  Some common area units are:%
		%
		\begin{itemize}
		\item square inches (in\(^2\))%
		\item square miles (mi\(^2\))%
		\item square meters (m\(^2\))%
		\end{itemize}
			\newpage
			
		\begin{ex}
			Give an example of something that could be measured using each of the units above%
		\end{ex}
			\vs{1}
			
		\begin{ex}
			A new house has a 2.5 acre lot.  What is the square footage of the lot?  Use the fact that there are 3 feet in 1 yard, and 4,840 square yards in 1 acre.%
		\end{ex}
			\vs{1}
			
	\subsection*{Measuring Volume}
		Volume is a measure of three-dimensional objects, the three-dimensional analogue of area.%
		
		\begin{rmk}[Volume Formulas for Common Shapes]
			\tabitem Rectangular prism: Length\(\cdot\)Width\(\cdot\)Height\\[15pt]
			\tabitem Sphere: \(\dfrac{4\pi}{3}\cdot\)Radius\(^3\)
		\end{rmk}
		
		\begin{ex}
			A new piece of luggage has a carrying capacity of 5.2 cubic feet of storage space.  How many cubic inches of space is there?%
		\end{ex}
			\vs{1}
			\newpage
			
		Liquid volume is often measured with special units.  Some of these are:%
		
		\begin{itemize}
			\item Milliliters (mL)%
			\item Liters (L)%
			\item Fluid Ounces (oz)%
			\item Cups (c)%
			\item Quarts (qt)%
			\item Gallons (gal)%
		\end{itemize}
		
		\begin{rmk}[Relationships Between Units]
			\tabitem There are 1000 milliliters in 1 liter\\[15pt]
			\tabitem There are 8 fluid ounces in 1 cup\\[15pt]
			\tabitem There are 4 cups in 1 quart\\[15pt]
			\tabitem There are 4 quarts in 1 gallon
		\end{rmk}
		
		\begin{ex}
			How many fluid ounces are in 2 gallons?%
		\end{ex}
			\vs{1}
			\newpage
			
	\subsection*{Measuring Temperature}
		There are two primary units used to measure temperature: Fahrenheit and Celsius%
		
		\begin{rmk}[Converting Between Fahrenheit and Celsius]
			To convert from Fahrenheit to Celsius, use the formula%
			\[^\circ C = (^\circ F-32)\cdot \dfrac{5}{9}\]
			To convert from Celsius to Fahrenheit, use the formula%
			\[^\circ F = ^\circ C\cdot \dfrac{9}{5} + 32\]
			
			A decent estimate to go from Celsius to Fahrenheit is to double the Celsius temperature, then add 32.%
		\end{rmk}
		
		\begin{ex}
			The average high in Norman in October is 75\(^\circ F\).  How high is that in Celsius?%
		\end{ex}
			\vs{1}
			
		\begin{ex}
			The 2022 Men's World Cup was moved from Summer 2022 to Winter 2022 because the average temperature in Qatar is about 41\(^\circ C\).  About how hot is that in Fahrenheit?%
		\end{ex}
			\vs{1}
\clearpage
\end{document}