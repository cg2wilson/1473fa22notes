\documentclass[notes]{subfiles}
\begin{document}
	\addcontentsline{toc}{section}{Quantitative Data}
	\refstepcounter{section}
	\fancyhead[RO,LE]{\bfseries \large \nameref{quantdata}} 
	\fancyhead[LO,RE]{\bfseries \currentname}
	\fancyfoot[C]{{}}
	\fancyfoot[RO,LE]{\large \thepage}	%Footer on Right \thepage is pagenumber
	\fancyfoot[LO,RE]{\large Unit 1}

\section*{Quantitative Data}\label{quantdata}
	\subsection*{Quantitative Data}	
		In order to represent quantitative data, we can use a frequency table just as we could for qualitative data.

		\begin{ex}
			The results for a math quiz are given below.
			{
				\begin{center}
					\begin{tabular}{lllllllll}
						7.5&7&9.5&9&10&9&7.5&5&9.5 \\
						10&9.5&9&7&7.5&10&8&7.5&9
					\end{tabular}
				\end{center}
			}
			Construct a frequency table for the data above, then construct a bar graph from the frequency table.
		\end{ex}
			\vs{2}

		\begin{question}
			What are some graphical concerns with the bar graph from the previous example?  What are some ways we could improve the graph?
		\end{question}
			\vs{1}
			\newpage

		\begin{defn}[Histogram]
			A \textbf{histogram} is a graphical representation of numerical data that is similar to a bar graph, but uses a number line as the horizontal axis.
		\end{defn}

		\begin{ex}
			Create the histogram for the previous example.
		\end{ex}
			\vs{2}

		With quantitative data, often we have to work with large input ranges.  We can handle large ranges by using \textbf{class intervals}.

		\begin{defn}[Class Intervals]
			\textbf{Class intervals} are groupings of input data used to make histograms are easier to read
		\end{defn}

		\begin{rmk}[Class Interval Rules]
			When creating class intervals, we have several rules.\\[15pt]

			\tabitem Each piece of data must fall into one of the classes\\[15pt]
			\tabitem Classes must not overlap\\[15pt]
			\tabitem Classes must be of equal width\\[15pt]
			\tabitem There must be no gaps between classes, even if a class has no data
		\end{rmk}
			\newpage

		\begin{ex}
			The height (in inches) of the members of a high school basketball team are collected in the table below:
			\begin{center}
				\begin{tabular}{lllll}
					65&70&68&72&71\\
					70&77&72&69&71
				\end{tabular}
			\end{center}
			\begin{enumerate}[(a)]
				\item Create four class intervals using the data.%
					\vs{1}

				\item Create the relative frequency distribution for the class intervals.%
					\vs{1}

				\item Use part (b) to create the histogram for the data.% 
					\vs{2}
			\end{enumerate}
		\end{ex}
			\newpage

		\begin{ex}
			Get into groups (your instructor will tell you the size).%
			\begin{enumerate}[(a)]
				\item Within your group, figure out how many countries each person has been to.%
					\vs{1}

				\item Your instructor will gather the information from each group and write it down for the class.  With your group, create the histogram for the data.%
					\vs{2}

			\end{enumerate}
		\end{ex}
			\newpage

		\begin{ex}
			The following table presents the number of hours worked last week by employees at a local drug store.
			\begin{center}
				\begin{tabular}{llllllll}
					52&18&2&20&9&9&11&6\\
					4&12&9&16&10&37&15&18\\
					4&3&17&19&12&20&11&14\\
					21&36&17&3&23&28&19&20
				\end{tabular}
			\end{center}
			Create the hisogram by breaking the data into 5 classes.
		\end{ex}
			\vs{1}
		
		Data can also be visualized by using another kind of graphic, called a \textbf{stem-and-leaf plot}.

		\begin{defn}[Stem-and-Leaf Plot]
			A \textbf{stem-and-leaf plot} is a graphic which separates numerical data into two pieces: the \textbf{stem} (such as the left-most digits) and the \textbf{leaf} (such as the right-most digit).%
		\end{defn}
			\newpage

		\begin{ex}
			Create the stem-and-leaf plot for the previous example.  Here's the data again:\\
			
			The following table presents the number of hours worked last week by employees at a local drug store.
			\begin{center}
				\begin{tabular}{llllllll}
					52&18&2&20&9&9&11&6\\
					4&12&9&16&10&37&15&18\\
					4&3&17&19&12&20&11&14\\
					21&36&17&3&23&28&19&20
				\end{tabular}
			\end{center}
		\end{ex}
			\vs{1}

		\begin{ex}
			A Spotify playlist purports to collect the 50 top songs of the week.  For the week of July 25 - July 29, the BPM (beats per minute) of those top 50 songs were collected and are listed below.
			\begin{center}
				\begin{tabular}{llllllllll}
					101&140&109&174&160&107&107&141&109&107\\
					132&169&103&142&126&147&154&166&115&103\\
					81&115&156&102&129&102&108&165&170&78\\
					122&95&170&84&112&186&118&180&125&176\\
					94&120&135&113&101&93&158&117&81&161
				\end{tabular}
			\end{center}
			\begin{enumerate}[(a)]
				\item Create a histogram for the data by dividing into classes of your choosing.%
					\vs{1}
					\newpage

				\item Create a stem-and-leaf plot for the data.%
					\vs{1}

				\item Which graphical representation (if either) do you think best displays the data?  What do you think makes it better than the other? Or, why do you think both representations are good representations?%
					\vs{1}

			\end{enumerate}
		\end{ex}
\clearpage
\end{document}